\documentclass[fontsize=14pt]{scrartcl}
\setkomafont{disposition}{\normalfont\bfseries}

\usepackage{amsthm}
\usepackage{amsmath}
\usepackage{amssymb}
\usepackage{fancyhdr}
\usepackage{mathtools}
\usepackage{pdfrender}
\usepackage{mathrsfs}
\usepackage{array}
\usepackage{graphicx}

\usepackage[margin=1in]{geometry}

\theoremstyle{definition}
\newtheorem{problem-internal}{Problem}[section]
\pagestyle{fancy}
\fancyhf{}
\renewcommand{\headrulewidth}{0pt}
\renewcommand*{\thefootnote}{\fnsymbol{footnote}}
\newcolumntype{L}{>{$}l<{$}}
\newcolumntype{C}{>{$}c<{$}}
\newcolumntype{R}{>{$}r<{$}}
\newcommand{\lcm}{\textup{lcm}}
\newcommand{\im}{\textup{im}}

\newenvironment{problem}{
\medskip
\begin{problem-internal}
}{
\end{problem-internal}
}

\newenvironment{solution}{
\begin{proof}[Solution]
\vspace{-8px}
\setlength{\parskip}{4px}
\setlength{\parindent}{0px}
}{
\end{proof}
}


\rhead{Tommy O'Shaughnessy\\
Steve Awodey\\
Category Theory\\
}
\begin{document}

\section*{Chapter 1}

\setcounter{section}{9}
\setcounter{problem-internal}{0}
\begin{problem}
    The objects of $\mathbf{R e l}$ are sets, and an arrow $A \rightarrow B$ is a relation from $A$ to $B$, that is, a subset $R \subseteq A \times B .$ The equality relation $\{\langle a, a\rangle \in A \times A \mid a \in A\}$ is the identity arrow on a set $A .$ Composition in Rel is to be given by
    $$
    S \circ R=\{\langle a, c\rangle \in A \times C \mid \exists b(\langle a, b\rangle \in R \&\langle b, c\rangle \in S)\}
    $$
    for $R \subseteq A \times B$ and $S \subseteq B \times C$.
\end{problem}
\begin{solution}
(a) Show that Rel is a category.\\
\textit{Composability}:\\
Shown above.\\\\
\textit{Identity}:\\
Shown above.\\\\
\textit{Associativity}:\\
Let $D\in\mathbf{Rel}$ such that $A\xrightarrow{R} B \xrightarrow{S} C\xrightarrow{T} D$. Consider $\langle a,d\rangle\in T\circ(S\circ R)$. 
I need to show that $\langle a,d\rangle\in(T\circ S)\circ R$.
By composition, there exists a $c$ such that $\langle a,c\rangle\in S\circ R$ and
$\langle c,d\rangle\in T$. Again, by composition, 
there exists a $b$ such that $\langle a,b\rangle\in R$ and $\langle b,c\rangle\in S$.
Since $\langle c,d\rangle\in T$, I have that $\langle b,d\rangle\in T\circ S$.
Finally, since $\langle a,b\rangle\in R$, then $\langle a,d\rangle\in(T\circ S)\circ R$.\\
Showing $(T\circ S)\circ R\subset T\circ(S\circ R)$ is done symmetrically.\\
\\
\textit{Unit}:\\
Let $\langle a,b\rangle\in R$. By \textit{identity}, $\langle a,a\rangle\in 1_A$ and $\langle b,b\rangle\in 1_B$. \\
Hence, $\langle a,b\rangle\in R\circ 1_A, 1_B\circ R$. If $\langle a,b\rangle\in R\circ 1_A, 1_B\circ R$; then definitionally $\langle a,b\rangle\in R$.
This shows that $R\circ 1_A=R=1_B\circ R$.\\

(b) Show also that there is a functor $G:$ Sets $\rightarrow$ Rel taking objects to themselves and each function $f: A \rightarrow B$ to its graph,
$$
G(f)=\{\langle a, f(a)\rangle \in A \times B \mid a \in A\}
$$
I wish to show that 
\begin{gather*}
    G(f:A\rightarrow B) = G(f):G(A)\rightarrow G(B)
\end{gather*}
for any function $f$. By the definition of $G$
\begin{gather*}
    G(f):G(A)\rightarrow G(B) = G(f):A\rightarrow B.
\end{gather*}
By the definition of $G(f)$, it is clear that $G(f)\subseteq A\times B$.
Thus, $G(f):A\rightarrow B$ is an arrow in $\mathbf{Rel}$.\\
Now I wish to show that $G(1_A) = 1_{G(A)}$:
\begin{gather*}
    \begin{tabular}
        {RLR}
        G(1_A)  &= \{\langle a, 1_A(a)\rangle \in A \times A \mid a \in A\} &\\
                &= \{\langle a, a\rangle \in A \times A \mid a \in A\}      &(1_A(a)=a)\\
                &= 1_A                                                      &\\
                &= 1_{G(A)}                                                 &(G(A)=A)
    \end{tabular}
\end{gather*}
\\
Finally, I wish to show 
\begin{gather*}
G(fg)=\{\langle a, fg(a)\rangle \in A \times C \mid a \in A\}
\text{ and }\\
GfGg=\{\langle a, c\rangle \in A \times C \mid \exists b(\langle a, b\rangle \in Gg \&\langle b, c\rangle \in Gf)\}
\end{gather*}
where $A\xrightarrow{g} B \xrightarrow{f} C$ are any two arrows in \textbf{Sets}.
If either set is empty, then $A$ or $C$ is empty. In that case, both sets are empty and equal.
Let $\langle a,fg(a) \rangle \in G(fg)$. By taking $g(a)\in B$, 
it's clear that $\langle a,fg(a) \rangle \in GfGg$.\\
Now let $\langle a,c \rangle \in GfGg$. There then exists some $b\in B$ such that $\langle a,b \rangle\in Gg$.
By definition, $b=g(a)$ and hence $c=fg(a)$. Thus, $\langle a,c \rangle=\langle a,fg(a) \rangle\in G(fg)$.
\\
(c) Finally, show that there is a functor $K: \mathbf{R e l}^{\mathrm{Op}} \rightarrow$ Rel taking each relation $R \subseteq A \times B$ to its converse $R^{c} \subseteq B \times A,$ where,
$$
\langle a, b\rangle \in R^{c} \Leftrightarrow\langle b, a\rangle \in R
$$
Consider 
\begin{gather*}
    \begin{tabular}
        {RL}
    K(SR)   &=K(\{\langle a, c\rangle \in A \times C \mid \exists b(\langle a, b\rangle \in R \,\&\,\langle b, c\rangle \in S)\})\\
            &=\{\langle c, a\rangle \in C \times A \mid \exists b(\langle c, b\rangle \in S \,\&\,\langle b, a\rangle \in R)\}\\
            &=\{\langle c, a\rangle \in C \times A \mid \exists b(\langle c, b\rangle \in KS \,\&\,\langle b, a\rangle \in KR)\}\\
            &=KSKR.
    \end{tabular}
\end{gather*}
\end{solution}

\setcounter{section}{9}
\setcounter{problem-internal}{1}
\begin{problem}

\end{problem}
\begin{solution}

\end{solution}

\end{document}