\documentclass[fontsize=14pt]{scrartcl}
\setkomafont{disposition}{\normalfont\bfseries}

\usepackage{amsthm}
\usepackage{amsmath}
\usepackage{amssymb}
\usepackage{fancyhdr}
\usepackage{mathtools}
\usepackage{pdfrender}
\usepackage{mathrsfs}
\usepackage{array}
\usepackage{graphicx}

\usepackage[margin=1in]{geometry}

\theoremstyle{definition}
\newtheorem{problem-internal}{Problem}[section]
\pagestyle{fancy}
\fancyhf{}
\renewcommand{\headrulewidth}{0pt}
\renewcommand*{\thefootnote}{\fnsymbol{footnote}}
\newcolumntype{L}{>{$}l<{$}}
\newcolumntype{C}{>{$}c<{$}}
\newcolumntype{R}{>{$}r<{$}}
\newcommand{\lcm}{\textup{lcm}}
\newcommand{\im}{\textup{im}}

\newenvironment{problem}{
\medskip
\begin{problem-internal}
}{
\end{problem-internal}
}

\newenvironment{solution}{
\begin{proof}[Solution]
\vspace{-8px}
\setlength{\parskip}{4px}
\setlength{\parindent}{0px}
}{
\end{proof}
}


\rhead{Tommy O'Shaughnessy\\
Prof. Feng Luo\\
Abstract Algebra I\\
12/05/2019}
\begin{document}

\section*{Chapter 1}

\setcounter{section}{9}
\setcounter{problem-internal}{0}
\begin{problem}
    The objects of $\mathbf{R e l}$ are sets, and an arrow $A \rightarrow B$ is a relation from $A$ to $B$, that is, a subset $R \subseteq A \times B .$ The equality relation $\{\langle a, a\rangle \in A \times A \mid a \in A\}$ is the identity arrow on a set $A .$ Composition in Rel is to be given by
    $$
    S \circ R=\{\langle a, c\rangle \in A \times C \mid \exists b(\langle a, b\rangle \in R \&\langle b, c\rangle \in S)\}
    $$
    for $R \subseteq A \times B$ and $S \subseteq B \times C$.
\end{problem}
\begin{solution}
(a) Show that Rel is a category.\\
Composability:\\
Identity:\\
Associativity:\\
Unit:\\

(b) Show also that there is a functor $G:$ Sets $\rightarrow$ Rel taking objects to themselves and each function $f: A \rightarrow B$ to its graph,
$$
G(f)=\{\langle a, f(a)\rangle \in A \times B \mid a \in A\}
$$
(c) Finally, show that there is a functor $C: \mathbf{R e l}^{\mathrm{Op}} \rightarrow$ Rel taking each relation $R \subseteq A \times B$ to its converse $R^{c} \subseteq B \times A,$ where,
$$
\langle a, b\rangle \in R^{c} \Leftrightarrow\langle b, a\rangle \in R
$$

\end{solution}

\setcounter{section}{9}
\setcounter{problem-internal}{1}
\begin{problem}

\end{problem}
\begin{solution}

\end{solution}

\end{document}